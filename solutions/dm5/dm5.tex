\documentclass[french]{article}
\usepackage[T1]{fontenc}
\usepackage[utf8]{inputenc}
\usepackage{lmodern}
\usepackage[a4paper]{geometry}
\usepackage{babel}
\usepackage{amsmath}
\usepackage{amsfonts}
\usepackage{tcolorbox}
\usepackage{color}
\usepackage{breqn}

\begin{document}
	\title{DM5:  Pendule à point d’attache mobile}
	\author{Gabriel PEREIRA DE CARVALHO}
	\date{Dernière modification \today}
	
	\maketitle
	
	\subsection*{Exercice 1}
	
	On sait que $\mathcal{L} = T - V$.
	
	D'abord, calculons l'énergie potentiel et cinétique pour chacun des deux points.
	
	Le point de masse $M$ a pour coordonées $(x_1, y_1) = (x, 0) \implies (\dot{x_1},\dot{y_1}) = (\dot{x}, 0)$. On a
	
	\begin{align}
		\begin{cases}
		T_1 &= \frac{1}{2} M (\dot{x_1}^2 + \dot{y_1}^2) = \frac{M x^2}{2}\\
		V_1 &= mgy_1 = 0
		\end{cases}
	\end{align}
	
	Alors, le point de masse $m$ a pour coordonées 
	$$ (x_2, y_2) = (x + l\sin(\theta),\quad -l\cos(\theta)) \implies (\dot{x_2}, \dot{y_2}) = (\dot{x} + l\dot{\theta}\cos(\theta),\quad l\dot{\theta}\sin(\theta)).$$
	
	On a
	
	\begin{align}
	\begin{cases}
	T_2 &= \frac{1}{2} m (\dot{x_2}^2 + \dot{y_2}^2) = \frac{m \dot{x}^2}{2} + \frac{m}{2}(2\dot{x}l\dot{\theta}\cos(\theta) + l^2\dot{\theta}^2)\\
	V_2 &= mgy_2 = -mgl\cos(\theta)
	\end{cases}
	\end{align}
	
	Ce qui donne 
	$$ \mathcal{L} = (T_1 + T_2) - (V_1 + V_2) =  \frac{(M+m)\dot{x}^2}{2} + \frac{m}{2}(2\dot{x}l\dot{\theta}\cos(\theta) + l^2\dot{\theta}^2) + mgl\cos(\theta)$$
	
	\subsection*{Exercice 2}
	
	On utilise les équation d'Euler-Lagrange par rapport aux coordonées $x$ et $\theta$.
	
	D'abord, considerons l'équation
	
	\begin{equation}
		\frac{\partial}{\partial \theta} \mathcal{L} = \frac{d}{dt}\left( \frac{\partial}{\partial \dot{\theta}} \mathcal{L} \right).
	\end{equation}
	
	On calcule les dérivées
	
	\begin{align}
		\begin{cases}
		\frac{\partial}{\partial \theta} \mathcal{L} &= -m\dot{x} l \dot{\theta} \sin(\theta) - m g l \sin(\theta) = -ml\sin(\theta)(\dot{x} \dot{\theta} + g)\\
		\frac{\partial}{\partial \dot{\theta}} \mathcal{L} &= \frac{m}{2} (2\dot{x}l\cos(\theta) + 2l^2\dot{\theta}) \implies \frac{d}{dt}\left( \frac{\partial}{\partial \dot{\theta}} \mathcal{L} \right) = m\ddot{x}l\cos(\theta) - m\dot{x}l\dot{\theta}\sin(\theta) + ml^2\ddot{\theta}
		\end{cases}
	\end{align}
	
	d'où
	
	\begin{align}
	&-ml \sin(\theta)(\dot{x}\dot{\theta} + g) = ml(\ddot{x} \cos(\theta) - \dot{x}\dot{\theta} \sin(\theta) + l\ddot{\theta})\\
	\iff &\sin(\theta)(\dot{x}\dot{\theta} + g) = \dot{x}\dot{\theta} \sin(\theta) - \ddot{x} \cos(\theta) - l\ddot{\theta}\\
	\iff & \ddot{x} \cos(\theta) + l\ddot{\theta} + \sin(\theta)g = 0.
	\end{align}
	
	Donc, on a notre première équation de mouvement. Alors, considerons l'équation
	
	\begin{equation}
		\frac{\partial}{\partial x} \mathcal{L} = \frac{d}{dt}\left( \frac{\partial}{\partial \dot{x}} \mathcal{L} \right).
	\end{equation}
	
	On calcule les dérivées
	
	\begin{align}
	\begin{cases}
	\frac{\partial}{\partial x} \mathcal{L} &= 0\\
	\frac{\partial}{\partial \dot{\theta}} \mathcal{L} &= (M+m)\dot{x} + ml\dot{\theta}\cos(\theta)  \implies \frac{d}{dt}\left( \frac{\partial}{\partial \dot{\theta}} \mathcal{L} \right) = (M+m)\ddot{x} + ml\ddot{\theta}\cos(\theta) - ml\dot{\theta}^2\sin(\theta)
	\end{cases}
	\end{align}
	
	d'où
	
	\begin{align}
	& 0 = (M+m)\ddot{x} + ml\ddot{\theta}\cos(\theta) - ml\dot{\theta}^2\sin(\theta)\\
	\iff &  (M+m)\ddot{x} + ml\ddot{\theta}\cos(\theta) - ml\dot{\theta}^2\sin(\theta) = 0.
	\end{align}
	
	Donc, on a deux équations de mouvement
	
	\begin{align}
		\begin{cases}
		\ddot{x} \cos(\theta) + l\ddot{\theta} + \sin(\theta)g &= 0\\
		(M+m)\ddot{x} + ml\ddot{\theta}\cos(\theta) - ml\dot{\theta}^2\sin(\theta) &= 0.
		\end{cases}
	\end{align}
	
	
	\subsection*{Exercice 3}
	
	En supposant $\theta << 1$, on a $\cos(\theta) \approx 1$ et $\sin(\theta) \approx \theta$ ce qui donne
	
	\begin{align}
	\begin{cases}
	\ddot{x} + l\ddot{\theta} + \theta g &= 0\\
	(M+m)\ddot{x} + ml\ddot{\theta} - ml\dot{\theta}^2\theta &= 0.
	\end{cases}
	\end{align}
	
	On remarque que le terme $ml\dot{\theta}^2\theta$ d'ordre 3 est négligeable pour des petites oscillations. On peut donc récrire les équations de mouvement
	
	\begin{align}
	\begin{cases}
	\ddot{x} + l\ddot{\theta} + \theta g &= 0\\
	(M+m)\ddot{x} + ml\ddot{\theta} &= 0.
	\end{cases}
	\end{align}
	
	En realisant la substitution $\ddot{x} = -l\ddot{\theta} -\theta g$, on a
	
	$$ Ml\ddot{\theta} + (M+m)\theta g  = 0 \iff \ddot{\theta} = -\left(\frac{m + M}{M}\right)\frac{g}{l}\theta$$
	
	on remarque que c'est analogue à l'oscillateur harmonique avec $\omega^2 = \left(\frac{m + M}{M}\right)\frac{g}{l}$. Donc
	
	\begin{equation}
		\theta = \theta_0 \cos(\omega t + \phi_0) \quad \text{avec $\theta_0, \phi_0$ les conditions initiales}
	\end{equation}
	
	Alors, $\ddot{x} = -\frac{ml}{M+m} \ddot{\theta}$ d'où
	
	\begin{align}
		&\ddot{x} = \frac{ml}{M+m} \omega^2 \theta_0 \cos(\omega t + \phi_0) \\
		\iff & \dot{x} = \frac{ml}{M+m} \omega \theta_0 \sin(\omega t + \phi_0) + C_1 \\
		\iff & x = -\frac{ml}{M+m} \theta_0 \cos(\omega t + \phi_0) + C_1t + C_2
	\end{align}
	
	avec $C_1, C_2$ les conditions initiales.
	
	
\end{document}