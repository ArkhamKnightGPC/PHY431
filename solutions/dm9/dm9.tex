\documentclass[french]{article}
\usepackage[T1]{fontenc}
\usepackage[utf8]{inputenc}
\usepackage{lmodern}
\usepackage[a4paper]{geometry}
\usepackage{babel}
\usepackage{amsmath}
\usepackage{amsfonts}
\usepackage{mathtools}
\usepackage{tcolorbox}
\usepackage{color}
\usepackage{breqn}

\begin{document}
	\title{DM9: Particule relativiste dans un champ électromagnétique}
	\author{Gabriel PEREIRA DE CARVALHO}
	\date{\today}
	
	\maketitle
	
	\subsection*{Exercice 1}
	
	Par définition, l'action le long d'une trajectoire est donnée par
	
	\begin{equation}
		\mathcal{S} = \int_{t_1}^{t_1} \mathcal{L}(x,\dot{x};t) dt
	\end{equation}
	
	D'où le Lagrangien d'interaction électromagnétique est donné par
	
	\begin{equation}
		\mathcal{L}_{emg} = -q \eta_{\mu \nu} A^{\mu} \dot{x}^{\nu}
	\end{equation}
	
	En développant la notation d'Einstein, on a
	
	\begin{align}
		\eta_{\mu \nu} A^{\mu} \dot{x}^{\nu} &= A^0 \dot{x}^0 - A^1 \dot{x}^1 - A^2 \dot{x}^2 - A^3 \dot{x}^3 \\
		&= \frac{\phi}{c} c \frac{dt}{dt} - A_x \frac{dx}{dt} - A_y \frac{dy}{dt} - A_z \frac{dz}{dt} \\
		&= \phi - A_x v_x - A_y v_y - A_z v_z \\
		&= \phi - \vec{v} \cdot \vec{A}
	\end{align}
	
	On en conclue que
	
	\begin{equation}
	\mathcal{L}_{emg} = q \left(-\phi + \vec{v} \cdot \vec{A} \right)
	\end{equation}
	
	\subsection*{Exercice 2}
	
	Le lagrangien d'une particule libre est donné par
	
	\begin{equation}
		\mathcal{L}_{lib} = \frac{1}{2} m \dot{x}
	\end{equation}
	
	d'où on a le Lagrangien total
	
	\begin{equation}
		\mathcal{L} = \mathcal{L}_{lib} +\mathcal{L}_{emg}  = \frac{1}{2} m \dot{x} + q \left(-\phi + \vec{v} \cdot \vec{A} \right)
	\end{equation}
	
	Alors, calculons les moments conjugués $p_i$. On a
	
	\begin{align}
		p_i &= \frac{\partial \mathcal{L}}{\partial \dot{x}_i} \\
		&= \frac{\partial}{\partial \dot{x}_i} \left( \frac{1}{2} m \dot{x} + q \left(-\phi + \vec{v} \cdot \vec{A} \right) \right) \\
		&= m \dot{x}_i + q A_i
	\end{align}
	
	On en conclue que $\vec{p} = \frac{\partial \mathcal{L}}{\partial \dot{x}} = m \dot{x} + q \vec{A}$.
	
	\subsection*{Exercice 3}
	
	Écrivons l'équation d'Euler Lagrange
	
	\begin{equation}
		\frac{d}{d t} \left(\frac{\partial \mathcal{L}}{\partial \dot{x}}\right) = \frac{\partial \mathcal{L}}{\partial x} \iff \frac{d}{dt} \left( m\dot{x} + q \vec{A} \right) = q \left( - \nabla \phi + \nabla (\vec{v} \cdot \vec{A}) \right)
	\end{equation}
	
	en fonction de $\vec{p}_{lib}$, on a
	
	\begin{equation}
		\frac{d}{dt} \vec{p}_{lib} = q \left( -\frac{d\vec{A}}{dt} - \nabla \phi + \nabla (\vec{v} \cdot \vec{A}) \right)
	\end{equation}
	
	\subsection*{Exercice 4}
	
	On calcule
	
	\begin{align}
		\frac{d \vec{A}}{dt}  &= \frac{\partial \vec{A}}{\partial t} + \left(\dot{x}\frac{\partial \vec{A}}{\partial x} + \dot{y}\frac{\partial \vec{A}}{\partial y} + \dot{z}\frac{\partial \vec{A}}{\partial z} \right)\\
		&= \frac{\partial \vec{A}}{\partial t} + (\vec{v} \cdot \nabla) \vec{A}
	\end{align}
	
	Par la relation $\nabla (\vec{v} \cdot \vec{A}) = (\vec{v} \cdot \nabla) \vec{A} + \vec{v} \times (\nabla \times A)$ on a 
	
	\begin{equation}
		-\frac{d \vec{A}}{dt} + \nabla (\vec{v} \cdot \vec{A}) = - \frac{\partial \vec{A}}{\partial t} + \vec{v} \times (\nabla \times A)
	\end{equation}
	
	Donc, on peut récrire l'équation d'Euler Lagrange 
	
	\begin{equation}
		\frac{d}{dt} \vec{p}_{lib} = q \left( - \nabla \phi  - \frac{\partial \vec{A}}{\partial t} + \vec{v} \times (\nabla \times A)  \right)
	\end{equation}
	
	Alors, on utilise les rélations $\vec{B} = \nabla \times A$ et $\vec{E} =  - \nabla \phi  - \frac{\partial \vec{A}}{\partial t}$ qui donnent
	
	\begin{equation}
		\frac{d}{dt} \vec{p}_{lib} = q \left( \vec{E} + v \times \vec{B} \right)
	\end{equation}
	
	
\end{document}
	