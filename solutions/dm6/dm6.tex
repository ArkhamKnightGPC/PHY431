\documentclass[french]{article}
\usepackage[T1]{fontenc}
\usepackage[utf8]{inputenc}
\usepackage{lmodern}
\usepackage[a4paper]{geometry}
\usepackage{babel}
\usepackage{amsmath}
\usepackage{amsfonts}
\usepackage{tcolorbox}
\usepackage{color}
\usepackage{breqn}

\begin{document}
	\title{DM6: Comparaison entre les énergies de fission des noyaux atomiques et de dissociation des molécules}
	\author{Gabriel PEREIRA DE CARVALHO}
	\date{\today}
	
	\maketitle
	
	\subsection*{Exercice 1}
	
	On remarque que l'énergie de liaison totale est plus élevé pour les produits de fission
	
	\begin{align} 
	\begin{cases}
		E_{liaison}^{initial} &= 235 \cdot 7,7 \mathrm{MeV} = 1809,5 \mathrm{MeV}\\
		E_{liaison}^{final} &= 92 \cdot 8,7 \mathrm{MeV} + 142 \cdot 8,5 \mathrm{MeV}  = 2007,4 \mathrm{MeV}
	\end{cases}
	\end{align}
	
	or on a $E_{\mathrm{libérée}} = -197,9 \mathrm{MeV}$.
	
	On conclue que la réaction est \textbf{exothermique}.
	
	\subsection*{Exercice 2}
	
	L'énergie produite par molécule de $\text{\textit{BeH}}^{++}$ est
	
	$$ \frac{47 \mathrm{kcal}}{\mathrm{1 mole}} \times \frac{4,18 \cdot 10^2 \mathrm{J}}{1 \mathrm{kcal}} \times \frac{6,24 \cdot 10^{12} \mathrm{MeV} }{1 \mathrm{J}} \times \frac{1 \mathrm{mole}}{6,02 \cdot 10^{23}\mathrm{molécules}} = 7,31 \cdot 10^{-7} \frac{\mathrm{MeV}}{\mathrm{molécule}} $$
	
	donc l'énergie de fission est $2,7 \cdot 10^8$ fois plus grand. 
	
\end{document}