\documentclass[french]{article}
\usepackage[T1]{fontenc}
\usepackage[utf8]{inputenc}
\usepackage{lmodern}
\usepackage[a4paper]{geometry}
\usepackage{babel}
\usepackage{amsmath}
\usepackage{amsfonts}
\usepackage{mathtools}
\usepackage{tcolorbox}
\usepackage{color}
\usepackage{breqn}

\begin{document}
	\title{DM8: Pendule double}
	\author{Gabriel PEREIRA DE CARVALHO}
	\date{\today}
	
	\maketitle
	
	\subsection*{Exercice 1}
	
	\subsubsection*{Partie 1: Calculer le Lagrangien}
	
	D'abord, écrivons les coordonées de chaque masse.
	
	\begin{align}
		P_1 &= (x_1, y_1) = (l_1 \sin(\theta_1), l_1 \cos(\theta_1))\\
		P_2 &= (x_2, y_2) = (l_1 \sin(\theta_1) + l_2 \sin(\theta_2), l_1 \cos(\theta_1) + l_2 \cos(\theta_2)).
	\end{align}
	
	On sait que le Lagrangien est donné par $\mathcal{L}(\theta_1, \theta_2; \dot{\theta}_1, \dot{\theta}_2) = (T_1 + T_2) - (V_1 + V_2)$. Alors, écrivons les énergies cinétique et potentiel de chaque masse.
	
	Pour la particule 1, on a
	
	\begin{align}
		\begin{cases}
		T_1 &= \frac{m_1(\dot{x}_1^2 + \dot{y}_1^2)}{2} = \frac{m_1}{2}\left[l_1^2 \dot{\theta}_1^2 \cos^2(\theta_1) + l_2^2 \dot{\theta}_2^2 \sin^2(\theta_2)\right] = \frac{m_1 l_1^2 \dot{\theta}_1^2}{2}\\
		V_1 &= -m_1gy_1 = -m_1gl_1\cos(\theta_1)
		\end{cases}
	\end{align}
	
	et pour la particule 2
	
	\begin{align}
		\begin{cases}
		T_2 &= \frac{m_1(\dot{x}_2^2 + \dot{y}_2^2)}{2} = \frac{m_2}{2} \left[ (l_1\dot{\theta}_1\cos(\theta_1) + l_2\dot{\theta}_2\cos(\theta_2))^2 + (l_1\dot{\theta}_1\sin(\theta_1) + l_2 \dot{\theta}_2 \sin(\theta_2))^2\right] \\
		&=  \frac{m_2}{2} \left[ l_1^2\dot{\theta}_1^2 + 2l_1l_2\dot{\theta}_1\dot{\theta}_2\cos(\theta_1 - \theta_2) + l_2^2 \dot{\theta}_2^2 \right] \\
		V_2 &= -m_2gy_2 = -m_2g(l_1\cos(\theta_1) + l_2\cos(\theta_2))
		\end{cases}
	\end{align}
	
	alors, on a le Lagrangien
	
	\begin{equation}
		\mathcal{L} = \frac{(m_1+m_2)l_1^2 \dot{\theta}_1^2}{2} + (m_1+m_2)gl_1\cos(\theta_1) + \frac{m_2 l_2^2 \dot{\theta}_2^2}{2} + m_2gl_2\cos(\theta_2) + m_2 l_1 l_2 \dot{\theta}_1\dot{\theta}_2\cos(\theta_1 - \theta_2)
	\end{equation}
	
	où on observe un terme de couplage entre les particules.
	
	\subsubsection*{Partie 2: Équations de mouvement}
	
	On utilise les équations d'Euler-Lagrange par rapport aux coordonées $\theta_1$ et $\theta_2$.
	
	D'abord, considerons l'équation
	\begin{equation}
		\frac{\partial}{\partial \theta_1} \mathcal{L} = \frac{d}{dt} \left(\frac{\partial}{\partial \dot{\theta}_1} \mathcal{L}\right)
	\end{equation}
	
	On calcule les dérivées
	
	\begin{align}
		\begin{cases}
			\frac{\partial}{\partial \theta_1} \mathcal{L} &= -(m_1 + m_2)gl_1\sin(\theta_1) - m_2l_1l_2\dot{\theta}_1\dot{\theta_2}\sin(\theta_1 - \theta_2)\\
			\frac{\partial}{\partial \dot{\theta}_1} \mathcal{L} &= (m_1 + m_2)l_1^2 \dot{\theta}_1 + m_2l_1l_2 \dot{\theta}_2\cos(\theta_1 - \theta_2) \\ \frac{d}{dt} \left(\frac{\partial}{\partial \dot{\theta}_1} \mathcal{L}\right) &= (m_1 + m_2)l_1^2\ddot{\theta}_1 + m_2l_1l_2\ddot{\theta_2}\cos(\theta_1 - \theta_2) - m_2l_1l_2\dot{\theta}_2(\dot{\theta_1} - \dot{\theta_2})\sin(\theta_1 - \theta_2)
		\end{cases}
	\end{align}
	
	
	Donc, on a notre première équation de mouvement
	
	$$ -(m_1 + m_2)gl_1\sin(\theta_1) - m_2l_1l_2\dot{\theta}_1\dot{\theta_2}\sin(\theta_1 - \theta_2) = (m_1 + m_2)l_1^2\ddot{\theta}_1 + m_2l_1l_2\ddot{\theta_2}\cos(\theta_1 - \theta_2) - m_2l_1l_2\dot{\theta}_2(\dot{\theta_1} - \dot{\theta_2})\sin(\theta_1 - \theta_2)$$
	
	Alors, considerons l'équation
	
	\begin{equation}
		\frac{\partial}{\partial \theta_2} \mathcal{L} = \frac{d}{dt} \left(\frac{\partial}{\partial \dot{\theta}_2} \mathcal{L}\right)
	\end{equation}
	
	On calcule les dérivées
	
	\begin{align}
	\begin{cases}
	\frac{\partial}{\partial \theta_2} \mathcal{L} &= - m_2gl_2\sin(\theta_2) + m_2l_1l_2\dot{\theta}_1\dot{\theta_2}\sin(\theta_1 - \theta_2)\\
	\frac{\partial}{\partial \dot{\theta}_2} \mathcal{L} &= m_2l_2^2\dot{\theta}_2 + m_2l_1l_2\dot{\theta}_1\cos(\theta_1 - \theta_2) \\ \frac{d}{dt} \left(\frac{\partial}{\partial \dot{\theta}_2} \mathcal{L}\right) &= m_2l_2^2\ddot{\theta}_2 + m_2l_1l_2\ddot{\theta_1}\cos(\theta_1 - \theta_2) - m_2l_1l_2\dot{\theta}_1(\dot{\theta_1} - \dot{\theta_2})\sin(\theta_1-\theta_2)
	\end{cases}
	\end{align}
	
	Donc, on a notre première deuxième équation de mouvement
	
	$$ -m_2gl_2\sin(\theta_2) + m_2l_1l_2\dot{\theta}_1\dot{\theta_2}\sin(\theta_1 - \theta_2) =   m_2l_2^2\ddot{\theta}_2 + m_2l_1l_2\ddot{\theta_1}\cos(\theta_1 - \theta_2) - m_2l_1l_2\dot{\theta}_1(\dot{\theta_1} - \dot{\theta_2})\sin(\theta_1-\theta_2)$$
	
	\begin{tcolorbox}[colback=blue!5!white,colframe=blue!75!black]
		\begingroup\makeatletter\def\f@size{5}\check@mathfonts
		\begin{align}
			\begin{cases}
			-(m_1 + m_2)gl_1\sin(\theta_1) - m_2l_1l_2\dot{\theta}_1\dot{\theta_2}\sin(\theta_1 - \theta_2) = (m_1 + m_2)l_1^2\ddot{\theta}_1 + m_2l_1l_2\ddot{\theta_2}\cos(\theta_1 - \theta_2) - m_2l_1l_2\dot{\theta}_2(\dot{\theta_1} - \dot{\theta_2})\sin(\theta_1 - \theta_2) \\
			-m_2gl_2\sin(\theta_2) + m_2l_1l_2\dot{\theta}_1\dot{\theta_2}\sin(\theta_1 - \theta_2) =   m_2l_2^2\ddot{\theta}_2 + m_2l_1l_2\ddot{\theta_1}\cos(\theta_1 - \theta_2) - m_2l_1l_2\dot{\theta}_1(\dot{\theta_1} - \dot{\theta_2})\sin(\theta_1-\theta_2)
			\end{cases}
		\end{align}
		\endgroup
	\end{tcolorbox}
	
	\subsection*{Exercice 2}
	
	On remarque que le lagrangien ne dépend pas explicitement du temps, donc le système est isolé et \textbf{l'énergie est conservée}. Cependant, l'impulsion n'est pas conservé car le lagrangien n'est pas invariant par translation et le moment cinétique est conservé seulement dans le axe perpendiculaire au plan du mouvement.
	
	Par le \textbf{Théorème de Poincaré-Bendixson}, une condition nécessaire pour un système chaotique est avoir $\geq 3$ variables indépendantes. Donc, on conclue que le double pendule décrit dans l'exercice n'est pas chaotique.
	
	Cependant, on remarque que si les masses ne restent pas dans un même plan vertical, on aurait plus de dégrés de liberté et là le système pourrait être chaotique.
	
\end{document}