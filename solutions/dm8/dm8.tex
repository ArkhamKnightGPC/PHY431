\documentclass[french]{article}
\usepackage[T1]{fontenc}
\usepackage[utf8]{inputenc}
\usepackage{lmodern}
\usepackage[a4paper]{geometry}
\usepackage{babel}
\usepackage{amsmath}
\usepackage{amsfonts}
\usepackage{mathtools}
\usepackage{tcolorbox}
\usepackage{color}
\usepackage{breqn}

\begin{document}
	\title{DM8: Pendule double}
	\author{Gabriel PEREIRA DE CARVALHO}
	\date{\today}
	
	\maketitle
	
	\subsection*{Exercice 1}
	
	\subsubsection*{Partie 1: Calculer le Lagrangien}
	
	D'abord, écrivons les coordonées de chaque masse.
	
	\begin{align}
		P_1 &= (x_1, y_1) = (l_1 \sin(\theta_1), l_1 \cos(\theta_1))\\
		P_2 &= (x_2, y_2) = (l_1 \sin(\theta_1) + l_2 \sin(\theta_2), l_1 \cos(\theta_1) + l_2 \cos(\theta_2)).
	\end{align}
	
	On sait que le Lagrangien est donné par $\mathcal{L}(\theta_1, \theta_2; \dot{\theta}_1, \dot{\theta}_2) = (T_1 + T_2) - (V_1 + V_2)$. Alors, écrivons les énergies cinétique et potentiel de chaque masse.
	
	Pour la particule 1, on a
	
	\begin{align}
		\begin{cases}
		T_1 &= \frac{m_1(\dot{x}_1^2 + \dot{y}_1^2)}{2} = \frac{m_1}{2}\left[l_1^2 \dot{\theta}_1^2 \cos^2(\theta_1) + l_2^2 \dot{\theta}_2^2 \sin^2(\theta_2)\right] = \frac{m_1 l_1^2 \dot{\theta}_1^2}{2}\\
		V_1 &= -m_1gy_1 = -m_1gl_1\cos(\theta_1)
		\end{cases}
	\end{align}
	
	et pour la particule 2
	
	\begin{align}
		\begin{cases}
		T_2 &= \frac{m_1(\dot{x}_2^2 + \dot{y}_2^2)}{2} = \frac{m_2}{2} \left[ (l_1\dot{\theta}_1\cos(\theta_1) + l_2\dot{\theta}_2\cos(\theta_2))^2 + (l_1\dot{\theta}_1\sin(\theta_1) + l_2 \dot{\theta}_2 \sin(\theta_2))^2\right] \\
		&=  \frac{m_2}{2} \left[ l_1^2\dot{\theta}_1^2 + 2l_1l_2\dot{\theta}_1\dot{\theta}_2\cos(\theta_1 - \theta_2) + l_2^2 \dot{\theta}_2^2 \right] \\
		V_2 &= -m_2gy_2 = -m_2g(l_1\cos(\theta_1) + l_2\cos(\theta_2))
		\end{cases}
	\end{align}
	
	alors, on a le Lagrangien
	
	\begin{equation}
		\mathcal{L} = \frac{(m_1+m_2)l_1^2 \dot{\theta}_1^2}{2} - (m_1+m_2)gl_1\cos(\theta_1) + \frac{m_2 l_2^2 \dot{\theta}_2^2}{2} - m_2gl_2\cos(\theta_2) + m_2 l_1 l_2 \dot{\theta}_1\dot{\theta}_2\cos(\theta_1 - \theta_2)
	\end{equation}
	
	où on observe un terme de couplage entre les particules.
	
	\subsubsection*{Partie 2: Équations de mouvement}
	
	
	\subsection*{Exercice 2}
\end{document}